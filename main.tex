\documentclass[12pt]{article}

\title{CS-131:Hw5 Fun with Sorting}
\author{Matthew McLaughlin}
\date{\today}


\setlength{\topmargin}{0.0in}
\setlength{\headheight}{0.0in}
\setlength{\headsep}{0.0in}
\setlength{\textheight}{9.0in}
\setlength{\oddsidemargin}{0in}
\setlength{\textwidth}{6.5in}

\usepackage{graphicx}
\usepackage[hidelinks]{hyperref}
\usepackage{enumerate}
\usepackage{amsmath}
\usepackage{amssymb}
\usepackage{listings}
%\usepackage{algorithm} 
%\usepackage{algorithmicx} 
%\usepackage{algpseudocode}
\usepackage[ruled,vlined]{algorithm2e}
\hypersetup{colorlinks=true,urlcolor=blue}
\lstset{
	frame=single,
	language=c,
	numbers=left,
}

\begin{document}

\maketitle
\tableofcontents

\subsubsection*{Assumptions And Notation}
I use one based indexing. $S$ is the input sequence
$S_{i,j}$ is the subsequence $S_i, \dots S_j$.
$P_i$ denotes the $i$th proccessor in the network.
$P_{i,j}$ is the $j$th value stored in proccessor $i$.
If $T$ is a list of values the $T \Leftarrow x$ adds $x$ to the end
of list $T$.
Let $M$ be the name of the network. 

\newpage

\section{Shear Sort}
For this problem I am assuming that sequence $S$ is of 
length $\sqrt{n} = k$ where $k$ is positive integer
representing the number of proccessors. 
This will make it possible to treat the sequence
like a square matrix by dividing the sequence into rows and assigning each proccessor to
a row. The assumption that $k$ is a factor of $n$ serves to simply the 
pseudocode and analysis. However, it does not affect the complexity of the problem. 
That is to say, with the assumption that $n$ is a square number and $n$ is divisible by $k$,
we could still in effect treat the sequence as a square matrix as before when $k*k = n$, yet
now we would need to simulate this with the procecssors we had avaiable. For example,
suppose $n=16$ and $k=2$. In this case we could have $P_1$ divide its portion of the 
sequence in half so that the first 4 values and last 4 are treated as two different subsequences
allowing $P_1$ to simulate the effect of having an extra proccessor.   

\subsection{Implementation}
Below I outline the procedure \textit{Parallel-shear-sort} for preforming shear sort
on sequence $S$. This procedure uses helper functions described below.
We are required to use a comparison exchange sorting algorithm. Assume
that the subroutine $\textit{sort}$ used by \textit{Parallel-shear-sort}
is mergesort and that it takes an additional argument specifying wether to sort
in increasing order or decreasing order \\
 
\begin{algorithm}[]
	\SetAlgoRefName{is-even}:
	\caption{outputs true if the agument $n$ is even}
	\SetAlgoLined
	\DontPrintSemicolon
	\KwIn{$n$ an integer}
	\KwResult{true if $n$ is even false otherwise}
	\textbf{return} $n$ mod $2 = 0$
\end{algorithm}

\begin{algorithm}[]
	\SetAlgoRefName{Distribute-sequence}:
	\caption{Stores a subsequence of $S$ in the proccessors}
	\SetAlgoLined
	\DontPrintSemicolon
	\LinesNumbered 
	\KwIn{A network of a collection of proccessor $P_1 \dots P_{k}$, Sequence $S$, $n$ is number of values in $S$}
	\KwResult{Each proccessor has a portion of $S$ in its memory.
	$P_{i}$ stores the subsequence $S_{a,b}$ where 
	$a=(i-1)(\sqrt{n}) + 1$ and $b= (i)\sqrt{n}$. Thus, the routine effectively
	transforms $S$ into a square matrix and distribute it amongst the 
	proccessors so that $P_i$ stores the $i$th row of the matrix}
	$i \leftarrow 1$ \;
	$s \leftarrow \sqrt{n}$ \;
	\While{$i \leq s$}{
		$a \leftarrow (i-1)s + 1$ \;
		$b \leftarrow i \cdot s$ \;
		$P_{i} \leftarrow S_{a,b}$ \;
		$i \leftarrow i + 1$
	}
\end{algorithm}


\begin{algorithm}[]
	\SetAlgoRefName{Transpose}:
	\SetAlgoLined
	\DontPrintSemicolon
	\LinesNumbered 
	\caption{Simulates a matrix transpose on the network}
	\KwIn{A network of a collection of proccessor $P_1 \dots P_{k}$, $m$ is the number
	of values that each proccessor keeps in memory ($m=sqrt{n}$ (see \textit{Parallel-shear-sort})}
	\KwResult{If we think of $M$ as a matrix where each proccessor
	$P_i$ in $M$ stores the $i$th row in $M$ then after this routine
	returns each $P_i$ will store the $i$th column of $M$}
	\ForEach{$P_i \in M$}{
		\For{$j \leftarrow 1$ to $m$}{
			\If{$j \neq i$}{
				$P_{j,i} \leftarrow P_{i,j}$
			}
		}
	}
\end{algorithm}
The crucial part of the \textit{Transpose} routine is line 4 where
the $i$th proccessor sends it's $i$th row to all of the other 
proccessors. The effect is that the old entry $i$th entry in $P_j$ is 
overwritten with $P_{i,i}$


\begin{algorithm}[]
	\SetAlgoRefName{Output-sequence}:
	\SetAlgoLined
	\DontPrintSemicolon
	\LinesNumbered 
	\caption{Output the sequence obtained by
	scanning over the network using snake like indices}
	\KwIn{A network of a collection of proccessor $P_1 \dots P_{k}$, $m$ is the number
	of values that each proccessor keeps in memory ($m=sqrt{n}$ see \textit{Parallel-shear-sort})}
	\KwOut{Produces a sequence $T$ from $M$ according to the following rules. If $i$ is even then
	scan $P_i$ from left to right otherwise scan from
	right to left}
	\ForEach{$P_i \in M$}{
		$T \leftarrow $ empty-list()
		\uIf{is-even(i)}{
			\For{$j \leftarrow 1$ to $m$}{
				$T \Leftarrow P_{i,j}$ \;
			}
		}
		\uElse{
			\For{$j \leftarrow m$ down-to $1$}{
				$T \Leftarrow P_{i,j}$ \;
			}
		}
	}
\end{algorithm}

\begin{algorithm}[]
	\SetAlgoRefName{Parallel-sort}:
	\SetAlgoLined
	\DontPrintSemicolon
	\LinesNumbered 
	\caption{Produces a sorted version of $S$}
	\KwIn{A network of a collection of proccessor $P_1 \dots P_{k}$, Sequence $S$,
	$n$ is the number of values in $S$}
	Distribute-sequence($M$,$S$,$n$) \;
	$cycle\_num \leftarrow 0$ \;
	$m \leftarrow \sqrt{n}$ \;
	$max\_cycles \leftarrow \log(m)$ \;
	\While{$cycle\_num \leq max\_cycles$}{
		\uIf{is-even($cycle\_num$)}{
			\ForEach{$P_i \in M$}{
				\uIf{is-even(i)}{
					sort($P_i$,\textit{increasing})\;
				}
				\uElse{
					sort($P_i$,\textit{decreasing})\;
				}
			}
		}
		\uElse{
			Transpose($M$,$m$)\;
			sort($P_i$,\textit{increasing})\;
			Transpose($M$,$m$)\;
		}
		$cycle\_num \leftarrow cycle\_num + 1$\;
	}
	Output-sequence($M$,$m$)\;
\end{algorithm}
\newpage

\subsection{Analysis}
\subsubsection{Amount of compare exchange operations performed by each proccessor} \label{exchange}
Recall that we use merge sort to preform compare exchange operations.
Thus, the total number of comparisons performed between all the proccessors 
in a given cycle is $O(n \log n)$. We know by the theorem stated in the slides that the number of
iterations until \textit{Parallel-shear-sort} converges is at most 
$O(\log\sqrt{n}) = O(\frac1{2}\log(n))$. 
This means the total number of compare exchange operations 
performed over \textit{all} cycles is at most \\
$(\frac1{2}\log n)(n \log n) = \frac{n}{2}\log^2(n)$

\subsubsection{Amount of memory needed by each proccessor}
Each proccessor needs to be able to store $\sqrt{n}$ items. In total $O(n)$ storage
is needed.

\subsubsection{Number of Cross-Bar communications cycles }
The only time communications between proccessors takes place is during the transpose
phase of \textit{Parallel-shear-sort} in lines $13,15$. This means only $\frac1{2}$
the time (during an odd cycle) communications will take place. \\ 

\noindent
Tranpose is invoked twice in the else block. Each inovations requires $n$ communications 
which mean there will be $2n$ during an odd cycle. \\

\noindent
In \ref{exchange} we showed that there are no more than $\frac1{2}\log n$
cycles.  \\

\noindent
Putting all these ideas together, we see the total number of cross bar communications performed by
the algorithm is $\ldots$ 
$$(\frac1{2})(2n)(\frac1{2}\log n) = \frac{n \log n}{2}$$

which is $O(n \log n)$
\newpage

\section{Bitonic Sort}
I assume the $n=2^j$ for some positive integer $j$ and that $n$ is divisible by $k$, the number of proccessor
I assume the existence of two function, \textit{send} and \textit{read} responsible for communication
between proccessors on the network. These functions are not defined 
explicitally but the headers are given. Also, note that \textit{Compare-exchange} is defined to work
when communication is between two different proccessors or just within the same proccesor.
\textit{Output-sequence} here simply scans over each proccessors memory starting at lower index
proccesors and going up outputing the contents of there memory. Since it is such a simple function
I have not included it.
Lastly, I assume that the input sequence $S$ is a bitonic sequence where the first half of the sequence 
is nondecreasing and the second half nonincreasing. This assumption was permitted by the instructor. 
Futhermore, even if it wasnt, it wouldnt be very hard to trasnform an arbitrary sequence so that 
it was bitonic

\subsection{Implementation}
\begin{algorithm}[]
	\SetAlgoRefName{send}
	\SetAlgoLined
	\DontPrintSemicolon
	\LinesNumbered 
	\caption{Sends data from one proccessor to another}
	\KwIn{The sending procceesor $P_i$, the element to be sent $x$, the receiving proccesor $P_j$}
	\KwResult{Element $x$ is sent from $P_i$ to $P_j$. Whatever is in $P_j$ read buffer is automatically
	overwritten with $x$}
\end{algorithm}

\begin{algorithm}[]
	\SetAlgoRefName{read}
	\SetAlgoLined
	\DontPrintSemicolon
	\LinesNumbered 
	\caption{procceesor $P_i$ reads its read buffer and returns whatever value is in there.}
	\KwIn{The proccessor whose read buffer is to be read}
	\KwResult{The value inside of $P_i$ read buffer. There is only a valid value here if 
	something was sent to $P_i$ from another proccessor. Otherwise the buffer is garbage}
\end{algorithm}

\begin{algorithm}[]
	\SetAlgoRefName{Distribute-sequence}
	\caption{Stores a subsequence of $S$ in the proccessors}
	\SetAlgoLined
	\DontPrintSemicolon
	\LinesNumbered 
	\KwIn{A network of a collection of proccessor $P_1 \dots P_{k}$, Sequence $S$, $n$ is number of values in $S$,
	$k$ is the number of proccessors}
	\KwResult{Calculates how many values to load into each proccessor based on $n$ then reads that many consecutive
	values into the correct processor}
	$i \leftarrow 1 $ \;
	$s \leftarrow \frac{n}{k}$ \;
	\While{$i \leq k$}{
		$a \leftarrow (i-1)(s) + 1$ \;
		$b \leftarrow i \cdot s$ \;
		$P_i \leftarrow S_{a,b}$ \;
		$i \leftarrow i + 1$ \;
	}
\end{algorithm}


\begin{algorithm}[]
	\SetAlgoRefName{Compare-exchange}
	\caption{Compares the elements between two processors and swaps them if the first proccessors 
	element is greater than the seconds}
	\SetAlgoLined
	\DontPrintSemicolon
	\LinesNumbered 
	\KwIn{A proccessor $P_i$, the index of the element for comparision in $P_i$ called $a$, 
		A second proccessor $P_j$, the index of the element for comparison in $P_j$ called $b$}
	\KwResult{We assume that $P_i$ should store smaller values and $P_j$ larger. Thus, $P_i$ and $P_j$
	send eachother values (if they are different proccesors) and $P_i$ keeps the minimum of the two values
	whereas $P_j$ keeps the maximum of the two}

	\uIf{$i \neq j$}{
		send($P_i,P_{i,a},P_j$) \;
		send($P_j,P_{j,b},P_i$)	\;
		$x \leftarrow $ read($P_i$) \;
		$P_{i,a} \leftarrow $ min($P_{i,a},x$)\;
		$x \leftarrow $ read($P_j$) \;
		$P_{j,b} \leftarrow $ min($P_{j,b},x$)\;
	}
	\uElse{
		$x \leftarrow P_{i,a}$\;
		\If{$x > P_{i,b}$}{
			$P_{i,a} \leftarrow P_{i,b}$\;
			$P_{i,b} \leftarrow x $ \;
		}
	}
\end{algorithm}

\begin{algorithm}[]
	\SetAlgoRefName{Single-proccessor-bitonic-sort}:
	\caption{Given a proccessor $P$ whose memory stores a bitoncic sequence.}
	\SetAlgoLined
	\DontPrintSemicolon
	\LinesNumbered 
	\KwIn{A single proccessor from the network $P_i$, $n$ number of proccessors in network that $P_i$ is from, $k$ number proccessor in network
	that $P_i$ is from}

	$\textit{list\_size} \leftarrow \frac{n}{k}$ \;
	$j \leftarrow 1$\;
	$\textit{offset} \leftarrow \textit{list\_size} \cdot 2^{-j}$\;
	\While{\textit{offset} $\geq 1$}{
		\For{$i = 1 $ to $(\textit{list\_size} \cdot 2^{-1})$}{
			Compare-exchange($P_i$, $i$, $P_i$, $i + \textit{offset}$)
		}
	}
\end{algorithm}

\begin{algorithm}[]
	\SetAlgoRefName{Parallel-bitonic-sort}:
	\caption{Produces a sorted sequence given a bitonic sequence $S$ where the first half of $S$ is nondecreasing 
	and the second half is nonincreasing}
	\SetAlgoLined
	\DontPrintSemicolon
	\LinesNumbered 
	\KwIn{A network of a collection of proccessor $P_1 \dots P_{k}$, Bitonic sequence $S$, $n$ is number of values in $S$,
	$k$ is the number of proccessors}
	Distribute-sequence($M$,$S$,$n$,$k$) \;
	$\textit{list\_size} \leftarrow \frac{n}{k}$ \;
	$j \leftarrow 1$\;
	$\textit{offset} \leftarrow \textit{list\_size} \cdot 2^{-j}$\;
	\While{\textit{offset} $\geq 1$}{
		\For{$i=1$ to $\frac{k}{2}$}{
			\For{$c = 1 $ to $(\textit{list\_size} \cdot 2^{-1})$}{
				$a \leftarrow i$ \;
				$b \leftarrow i + \textit{offset}$ \;
				Compare-exchange($P_a$, $c$, $P_b$, $c$)
			}
		}
	}
	\ForEach{$P_i$}{
		Single-procceesor-bitoncic-sort($P_i$)\;
	}
	Output-sequence($M$)\;

\end{algorithm}
\newpage 

\subsection{Analysis}
\subsubsection{Amount of compare exchange operations performed by each proccessor} \label{op}
There will be $\frac{n}{4}\log n $ iterations for the while loop in \textit{parallel-bitonic-sort}.
\textit{Single-proccesor-bitonic-sort} is called once for each proccessor after the while loop.
There are $n \log n$ total comparisons as a result of this function. 
Thus in total there are  $O(n \log n )$ communications amongst proccessors.

\subsubsection{Amount of memory needed by each proccessor}
A given proccessor will need to store $\frac{n}{k}$ elements. In total of course the storage is
$O(n)$

\subsubsection{Number of Cross-Bar communications cycles }
In this case the analaysis of the cross bar communication is idnetical to \ref{op}
thus there are $O(n \log n) $ cross bar communications cycles

\end{document}
